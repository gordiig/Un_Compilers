\newpage
	\section{Цель и задачи работы}
		\textbf{Цель работы}: приобретение практических навыков реализации таблично управляемых синтаксических анализаторов на
		примере анализатора операторного предшествования

		Задачи работы:
		\begin{enumerate}
			\item Ознакомиться с основными понятиями и определениями, лежащими в основе синтаксического анализа операторного предшествования.
			\item Изучить алгоритм синтаксического анализа операторного предшествования.
			\item Разработать, тестировать и отладить программу синтаксического анализа в соответствии с предложенным
				вариантом грамматики.
			\item Включить в программу синтаксического анализ семантические действия для реализации синтаксически
				управляемого перевода инфиксного выражения в обратную польскую нотацию.
		\end{enumerate}


%%%%%%%%%%%%%%%%%%%%%%%%%%%%%%
	\section{Листинг}
        
        \lstset{inputencoding=utf8x, extendedchars=\true, breaklines=true, numbers=left,
        keywordstyle=\color{blue}, commentstyle=\color{red}}
        
        \subsection{main.py}
        \lstinputlisting[language=python]{../../main.py}

        \subsection{analyzer.py}
        \lstinputlisting[language=python]{../../analyzer.py}


%%%%%%%%%%%%%%%%%%%%%%%%%%%%%%

	\section{Тесты}
		\begin{enumerate}
			\item $1 < 2 \rightarrow 12<$
			\item $3 <> 4 \rightarrow 34<>$
			\item $1 + 2 < 3 - 4 \rightarrow 12+34-<$
			\item $1 + 2 < 3 + 4 / 5 \rightarrow 12+345/+<$
			\item $1 + 2 < (3 + 4) / 5 \rightarrow 12+34+5/<$
			\item $(1 <> 3) < (2 + 3) / 4 \rightarrow 13<>23+4/<$
		\end{enumerate}


%%%%%%%%%%%%%%%%%%%%%%%%%%%%%%

	\newpage
	\section{Выводы}
	
	По результатам проведенной работы студент приобрел практические навыки в реализации алгоритма синтаксического анализатора
		операторного предшествования.

%%%%%%%%%%%%%%%%%%%%%%%%%%%%%%

	\section{Список литературы}
		\begin{enumerate}
			\item БЕЛОУСОВ А.И., ТКАЧЕВ С.Б. Дискретная математика: Учеб. Для вузов / Под ред. В.С. Зарубина, А.П. Крищенко. – М.: Изд-во МГТУ им. Н.Э. Баумана, 2001.
			\item АХО А., УЛЬМАН Дж. Теория синтаксического анализа, перевода и компиляции: В 2-х томах. Т.1.: Синтаксичечкий анализ. - М.: Мир, 1978.
			\item АХО А.В, ЛАМ М.С., СЕТИ Р., УЛЬМАН Дж.Д. Компиляторы: принципы, технологии и инструменты. – М.: Вильямс, 2008.
		\end{enumerate}
	