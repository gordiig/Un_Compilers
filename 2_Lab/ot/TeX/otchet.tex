\newpage
	\section{Цель и задачи работы}
		\textbf{Цель работы}: приобретение практических навыков реализации наиболее важных (но не всех) видов 
			преобразований грамматик, чтобы удовлетворить требованиям алгоритмов синтаксического разбора.\\

		\textbf{Задачи работы:}
		\begin{enumerate}
			\item Принять к сведению соглашения об обозначениях, принятые в литературе по теории формальных языков и 
				грамматик и кратко описанные в приложении.
			\item Познакомиться с основными понятиями и определениями теории формальных языков и грамматик.
			\item Детально разобраться в алгоритме устранения левой рекурсии.
			\item Разработать, тестировать и отладить программу устранения левой рекурсии.
			\item Разработать, тестировать и отладить программу преобразования грамматики в соответствии с 
				предложенным вариантом.
		\end{enumerate}


%%%%%%%%%%%%%%%%%%%%%%%%%%%%%%
	\section{Листинг}
        
        \lstset{inputencoding=utf8x, extendedchars=\true, breaklines=true, numbers=left,
        keywordstyle=\color{blue}, commentstyle=\color{red}}
        
        \subsection{main.py}
        \lstinputlisting[language=python]{../../main.py}

        \subsection{grammar.py}
        \lstinputlisting[language=python]{../../grammar.py}


%%%%%%%%%%%%%%%%%%%%%%%%%%%%%%

	\section{Тесты}
		\subsection{Устранение $\epsilon$ - правил}
			Вход:
			\begin{itemize}
				\item $S \rightarrow ABC $
				\item $A \rightarrow BB | \epsilon $
				\item $B \rightarrow CC | a $
				\item $C \rightarrow AA | b $
			\end{itemize}

			Выход:
			\begin{itemize}
				\item $S' \rightarrow S | \epsilon $
				\item $S \rightarrow B | ABC | A | C | BC | AC | AB $
				\item $A \rightarrow B | BB $
				\item $B \rightarrow a | CC | C $
				\item $C \rightarrow A | b | AA $
			\end{itemize}

			Вход:
			\begin{itemize}
				\item $S \rightarrow aSbS | bSaS | \epsilon $
			\end{itemize}

			Выход:
			\begin{itemize}
				\item $S' \rightarrow S | \epsilon $
				\item $S \rightarrow ba | aSb | bSa | bSaS | abS | baS | aSbS | ab $
			\end{itemize}

		\subsection{Устранение левой рекурсии}
			Вход:
			\begin{itemize}
				\item $S \rightarrow Aa | b $
				\item $A \rightarrow Ac | Sd | \epsilon$
			\end{itemize}

			Выход:
			\begin{itemize}
				\item $S \rightarrow Aa | b $
				\item $A \rightarrow bdA' | A' $
				\item $A' \rightarrow cA' | adA' | \epsilon $
			\end{itemize}

%%%%%%%%%%%%%%%%%%%%%%%%%%%%%%

	\newpage
	\section{Выводы}
	
	По результатам проведенной работы студент приобрел практические навыки в реализации наиболее 
		важных видов преобразований грамматик, чтобы удовлетворить требованиям алгоритмов синтаксического разбора.
	В том числе была реализована программа, принимающая грамматику, по которой строятся грамматики 
		без $\epsilon$ - правил и без левой рекурсии

%%%%%%%%%%%%%%%%%%%%%%%%%%%%%%

	\section{Список литературы}
		\begin{enumerate}
			\item БЕЛОУСОВ А.И., ТКАЧЕВ С.Б. Дискретная математика: Учеб. Для вузов / Под ред. В.С. Зарубина, А.П. Крищенко. – М.: Изд-во МГТУ им. Н.Э. Баумана, 2001.
			\item АХО А., УЛЬМАН Дж. Теория синтаксического анализа, перевода и компиляции: В 2-х томах. Т.1.: Синтаксичечкий анализ. - М.: Мир, 1978.
			\item АХО А.В, ЛАМ М.С., СЕТИ Р., УЛЬМАН Дж.Д. Компиляторы: принципы, технологии и инструменты. – М.: Вильямс, 2008.
		\end{enumerate}
	